% Created 2021-09-20 Mon 11:29
% Intended LaTeX compiler: pdflatex
\documentclass[11pt]{article}
\usepackage[utf8]{inputenc}
\usepackage[T1]{fontenc}
\usepackage{graphicx}
\usepackage{grffile}
\usepackage{longtable}
\usepackage{wrapfig}
\usepackage{rotating}
\usepackage[normalem]{ulem}
\usepackage{amsmath}
\usepackage{textcomp}
\usepackage{amssymb}
\usepackage{capt-of}
\usepackage{hyperref}
\date{\today}
\title{}
\hypersetup{
 pdfauthor={},
 pdftitle={},
 pdfkeywords={},
 pdfsubject={},
 pdfcreator={Emacs 27.2 (Org mode 9.4.4)}, 
 pdflang={English}}
\begin{document}

\tableofcontents

\section{Questão 1 - Comerciais de Energéticos}
\label{sec:org4b7dd1e}
\textbf{Contextualização}

Um estudo foi arquitetado para comparar comerciais da empresa
vedendora de energéricos, Red Bull. Neste estudo, mostrou-se a cada
participante os comerciais A e B, em ordem aleatória (A, B ou B,
A). Por fim, coletou-se a opinião de qual era melhor.

Haviam um total de 140 mulheres participantes, e 130 homens. O
comercial A foi selecionado por 65 mulheres e 67 homens.

\subsection{Encontre a Razão de Chances}
\label{sec:org1855af3}
\subsubsection{Comercial A, homens}
\label{sec:org468c52f}
Encontre a \texttt{Razão de Chances} de selecionar-se o comercial A, para
homens;
\subsubsection{Comercial A, mulheres}
\label{sec:org65784a2}
Encontre a \texttt{Razão de Chances} de selecionar-se o comercial A, para
mulheres.

\subsubsection{Comercial A, homens ou mulheres}
\label{sec:orgd9d8b79}
Encontre a \texttt{Razão de Chances} de se escolher o comercial A.


\subsection{Modelo Ajustado}
\label{sec:orga8aecc4}
\subsubsection{Parâmetros}
\label{sec:org8a59dea}
Encontre o Modelo Ajustado, estimando os valores \(b_0\) e \(b_1\).
\subsubsection{Razão de Chances do modelo}
\label{sec:org277fdbf}
Qual é a \texttt{Razão de Chances} estimada para um homem \((x=1)\) escolher o
comercial A, comparado à uma mulher \((x=0)\)?

\section{Questão 2 - Uso de áudio-visual em propagandas}
\label{sec:org1a0f98d}
\textbf{Contextualização}

Utilizaremos, nesse problema, dados de empresas de pequeno e médio
porte, quanto a seu \texttt{uso}, ou \texttt{não}, de áudio-visual em postagens em
mídias sociais.

\begin{center}
\begin{tabular}{lrrr}
\hline
 &  & Usam audio-visual & \\
\hline
Tamanho & Sim & Não & Total\\
\hline
Pequenas & 150 & 28 & 178\\
Médias & 27 & 25 & 52\\
Total & 177 & 53 & 230\\
\hline
\end{tabular}
\end{center}
\subsection{Proporções}
\label{sec:org2a73cbf}
\subsubsection{Proporção, pequenas empresas}
\label{sec:org6d399a9}
Qual proporção de pequenas empresas utiliza postagens áudio-visuais?
\subsubsection{Proporção, médias empresas}
\label{sec:orgeeab4cf}
Qual proporção de médias empresas utilizam postagens áudio-visuais?
\subsection{Razão de Chances}
\label{sec:org34c59ae}
\subsubsection{Pequenas empresas}
\label{sec:org20a43f8}
Encontre a \texttt{Razão de Chances} de empresas pequenas utilizarem A.V.
\subsubsection{Médias empresas}
\label{sec:org551babe}
Encontre a \texttt{Razão de Chances} de médias empresas utilizarem A.V.

\subsection{Modelo Ajustado}
\label{sec:org99ce477}
\subsubsection{Parâmetros}
\label{sec:orgd5d1426}
Encontre o Modelo Ajustado, estimando os valores \(b_0\) e \(b_1\).
\subsubsection{Razão de Chances do modelo}
\label{sec:org9cec49a}
Qual é a \texttt{Razão de Chances} estimada para uma pequena empresa \((x=1)\) escolher o
comercial A, comparado à uma média \((x=0)\)?


\section{Questão 3 - O Filme Será Lucrativo?}
\label{sec:org1dd0782}
\textbf{Contextualização:}

Ao se utilizar dados provindos de diversos filmes e vendas de
bilheterias, ajustou-se uma curva \texttt{Logística}, para tentar prever
quais filmes seriam lucrativos, ou não.

Temos um \texttt{Modelo Ajustado}, do tipo,
  \begin{equation}
\begin{aligned}
    \log\left(\dfrac{p}{1-p}\right) = \beta_0 + \beta_1{}x
\end{aligned}
\end{equation}

Em que \texttt{p} é a probabilidade de que um filme será lucrativo. E, \texttt{x}
é o log do lucro de bilheteria, na semana de estreia do filme (em
milhões de dólares).

Quando ajustado, o modelo encontrado foi:

\begin{equation}
  \begin{aligned}
    \log(\textrm{chances}) = b_0 + b_1 x = -1.41+0.781x
  \end{aligned}
\end{equation}

Percebe-se, pelo valor das \texttt{Chance}, \(e^{b_1}=2.184\), que caso o
lucro da semana estreial for \(e^1=2.71\) vezes maior, então o filme
será, em geral, \(2.2\) vezes mais lucrativo.

\subsection{Interpretação da Razão de Chances}
\label{sec:org13da156}
Dado que se aplicarmos uma \texttt{exponencial} à função de \texttt{modelo ajustado},
encontramos:

\begin{equation}
  \begin{aligned}
    \textrm{chances} = e^{-1.41+0.781x}=e^{-1.41}.e^{0.781x}
  \end{aligned}
\end{equation}
\subsubsection{Demonstre}
\label{sec:org353b7e1}
Demonstre que, para qualquer valor \(x\), ao se aumentar uma
unidade, a seguinte relação é válida,

\begin{equation}
  \begin{aligned}
    \frac{\textrm{chances}_{x+1}}{\textrm{chances}_{x}}
  \end{aligned}
\end{equation}

e vale \(e^{0.781}=2.184\). Essa análise justifica o resultado em \hyperref[sec:org277fdbf]{Razão de Chances do modelo}

\subsection{Decida se um filme será lucrativo!}
\label{sec:orgb0ff9c3}
Caso a semana inaugural tenha um lucro de

(a) \$20 milhões de dólares
(b) \$35 milhões de dólares
(c) \$50 milhões de dólares

o filme será lucrativo?
\end{document}