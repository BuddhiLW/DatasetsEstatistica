% Created 2021-10-21 Thu 15:54
% Intended LaTeX compiler: xelatex
\documentclass[12pt]{abntex2}
\usepackage{graphicx}
\usepackage{grffile}
\usepackage{longtable}
\usepackage{wrapfig}
\usepackage{rotating}
\usepackage[normalem]{ulem}
\usepackage{amsmath}
\usepackage{textcomp}
\usepackage{amssymb}
\usepackage{capt-of}
\usepackage{hyperref}
\hypersetup{colorlinks, allcolors=.,
 colorlinks=true,
 linkcolor={blue!78!white},
 urlcolor={purple},
 filecolor={winered},
}
\usepackage{xcolor} % to access the named colour LightGray
\definecolor{LightGray}{gray}{0.2}
\usepackage{minted}
\usemintedstyle{monokai}
\date{\today}
\title{}
\hypersetup{
 pdfauthor={},
 pdftitle={},
 pdfkeywords={},
 pdfsubject={},
 pdfcreator={Emacs 27.2 (Org mode 9.4.4)}, 
 pdflang={English}
}
\begin{document}

\tableofcontents
\clearpage


\chapter{Análise de Regressão Linear Múltipla, Conjuntos de Dados}
\label{sec:org395f2a7}
\section{\href{https://www.kaggle.com/augustus0498/life-expectancy-who}{Expectativa de Vida por País}}
\label{sec:orgd4e053d}
\begin{itemize}
\item \textbf{Modelar a variável expectativa de vida}
\end{itemize}
\subsection{Sobre o \texttt{Conjunto}}
\label{sec:orgfb4a40c}
\begin{itemize}
\item Dados de todos países, derivados da OMS.
\end{itemize}

\begin{itemize}
\item Dados:
\begin{itemize}
\item Country: país,
\item Year: ano correspondente aos dados,
\item Status: estado da industrialização,
\item Life expectancy: expectativa de vida,
\item Adult Mortality \(15<\text{idade}<60\): mortalidade adulta por mil habitantes,
\item Infant deaths: mortalidade infantil por mil habitantes,
\item Alcohol: uso de álcool por litro puro (+15 anos),
\item Percentage expenditure: gasto em saúde em percentagem do PIB per capta,
\item Hepatitis B: percentagem de imunização entre bebês com um ano,
\item Measles: quantidade de casos reportados de Rubeola por mil habitantes,
\item BMI: média de index de massa corpórea da população,
\item under-five deaths: quantidade de mortes antes dos cinco anos por mil habitantes,
\item Polio: quantidade de imunização entre bebês com um ano (percentagem),
\item Total expenditure: percentagem do gasto total do governo com saúde,
\item Diphtheria: percentagem de imunização de tétano (\(\leq 1 \text{ano}\)),
\item HIV/AIDS: morte a cada mil nascimentos (contanto nascimento de 0-4 anos de idade),
\item GDP: PIB em dólares,
\item Population: População,
\item thinness  1-19 years: prevalescência de pessoas magras (1-19 anos de idade),
\item thinness 5-9 years: prevalescência de pessoas magras (5-9 anos de idade),
\item Income composition of resources: IDH (qualidade de vida) em termos de composição por salário pessoal (provida por salário próprio),
\item Schooling: média do número de anos sendo educado (escolaridade).
\end{itemize}
\end{itemize}

\subsection{Como utilizar o conjunto (cópia de arquivo CSV, localmente)}
\label{sec:orge5233e5}
O \texttt{conjunto} apenas está disponível por meio da \href{https://drive.google.com/file/d/1nUnOwgAcWIqoyG0P5\_wnlh1wPxxSUqv5/view?usp=sharing}{cópia da núvem}, para o
computador local. Após baixar o arquivo \texttt{Life\_Expectancy\_Data.csv},
para seu computador, a análise deve ser feita, em relação a um
determinado ano, a qual foi indicado ao grupo.

Por exemplo, para a análise da expectativa de vida, para o ano de
2015, seria feita invocando os dados da seguinte maneira,

\begin{minted}[frame=lines,fontsize=\scriptsize,linenos=false, bgcolor=LightGray]{r}
life <- read.csv('./data/csv/Life_Expectancy_Data.csv')
life_2015 <- subset(life, Year=="2015")
life_2015[1:6,1:5] 
\end{minted}

\begin{verbatim}
   Country             Year Status     Life.expectancy Adult.Mortality
1  Afghanistan         2015 Developing 65.0            263            
17 Albania             2015 Developing 77.8             74            
33 Algeria             2015 Developing 75.6             19            
49 Angola              2015 Developing 52.4            335            
65 Antigua and Barbuda 2015 Developing 76.4             13            
81 Argentina           2015 Developing 76.3            116            
\end{verbatim}


Obs: apenas 5 colunas foram selecionadas, por motivos estéticos na formatação.

\section{\href{https://r-data.pmagunia.com/dataset/r-dataset-package-plm-cigar}{Consumo de Cigarro}}
\label{sec:org0d4e8c3}
\begin{itemize}
\item \textbf{Modelar quantidade de venda (sales)}.
\end{itemize}
\subsection{Sobre o \texttt{Conjunto}}
\label{sec:orgc64d986}
\begin{itemize}
\item Usado em:
\begin{itemize}
\item Baltagi, Badi H. (2001) Econometric Analysis of Panel Data, 2nd ed., John Wiley and Sons.
\item Baltagi, Badi H. (2013) Econometric Analysis of Panel Data, 5th ed., John Wiley and Sons.
\item Baltagi, B.H. and D. Levin (1992) “Cigarette taxation: raising revenues and reducing consumption”, Structural Changes and Economic Dynamics, 3(2), pp. 321–335.
\item Baltagi, B.H., J.M. Griffin and W. Xiong (2000) “To pool or not to pool: homogeneous versus heterogeneous estimators applied to cigarette demand”, Review of Economics and Statistics, 82(1), pp. 117–126.
\end{itemize}
\item Dados: anos 1963-1992
\begin{itemize}
\item State: estados norte americanos, \(\{1,2,3, \ldots, 51\}\).
\item Year: Anos 63-92.
\item Price: preço do maço.
\item Pop: população de fumantes.
\item Pop16: população acima de 16.
\item CPI: índice do preço (ideal) dado pelo consumidor.
\item NDI: Salário líquido per capta.
\item Sales: venda de cigarro por maço.
\item Pimin: preço mínimo de cigarro em estados vizinhos
\end{itemize}
\end{itemize}
\subsection{Como utilizar pelo R}
\label{sec:org6e2eb10}

O \texttt{conjunto de dados} se encontra sob o pacote \texttt{Ecdat} (Econometrics
Datasets [Conjuntos de Econometria]). Desta forma, precisamos
instalá-lo.

\begin{minted}[frame=lines,fontsize=\scriptsize,linenos=false, bgcolor=LightGray]{r}
install.packages("Ecdat",mirror="https://vps.fmvz.usp.br/CRAN/")
\end{minted}

Após instalação, precisamos invocar o pacote.
\begin{minted}[frame=lines,fontsize=\scriptsize,linenos=false, bgcolor=LightGray]{r}
load(Ecdat)
\end{minted}

Finalmente, podemos acessar o \texttt{conjunto}.
\begin{minted}[frame=lines,fontsize=\scriptsize,linenos=false, bgcolor=LightGray]{r}
cigarro <- data("Cigar")
cigarro
\end{minted}

\subsection{Invocação de conjuto, por arquivo local}
\label{sec:org458a125}

Dado que você copiou os conjunto(s) de dado(s) para um diretório
local, pode-se chamar o \texttt{conjunto} \href{https://drive.google.com/file/d/1iOQFp0TshV8km2X13fRkXGm8kd-ZVfSL/view?usp=sharing}{cigarro.csv }da seguinte maneira.

\begin{minted}[frame=lines,fontsize=\scriptsize,linenos=false, bgcolor=LightGray]{r}
  data <- read.csv('./data/csv/cigarro.csv')
  cigarro<- subset(data, year==90)
  head(data,5)
\end{minted}

\begin{verbatim}
    state year price pop     pop16   cpi   ndi   sales pimin
28  1     90   139.1  4129.2  3148.6 130.7 12806 108.6 132.3
58  3     90   130.2  3598.2  2703.7 130.7 13826  88.9 133.6
88  4     90   141.2  2411.1  1834.9 130.7 12370 113.1 129.1
118 5     90   163.8 29602.1 22490.6 130.7 17384  77.8 130.2
148 7     90   171.2  3242.9  2567.1 130.7 21447  91.5 150.9
\end{verbatim}

\section{\href{https://www.openintro.org/data/index.php?data=gifted}{Inteligência de prodígios}}
\label{sec:org99eddfe}
\begin{itemize}
\item \textbf{Modele o IQ da criança com as outras variáveis}
\end{itemize}
\subsection{Sobre o \texttt{Conjunto}}
\label{sec:org325f960}
\begin{itemize}
\item Referências:
\begin{itemize}
\item Graybill, F.A. \& Iyer, H.K., (1994) Regression Analysis: Concepts and Applications, Duxbury, p. 511-6.
\end{itemize}
\item Dados:
\begin{enumerate}
\item IQ da Criança.
\item IQ Pai.
\item IQ Mãe.
\item Período em meses, até primeiras palavras.
\item Período em meses, até quanto contou até dez.
\item Tempo passado, pelos pais, lendo livros, semanalmente.
\item Tempo passado assistindo programas educativos, semanalmente.
\item Tempo assistindo desenhos (\texttt{cartoons}), semanalmente.
\end{enumerate}
\end{itemize}

\subsection{Invocação de conjuto, por arquivo local}
\label{sec:orgdc56965}

Dado que você copiou os conjunto(s) de dado(s) para um diretório
local, pode-se chamar o \texttt{conjunto} \href{https://drive.google.com/file/d/1stukrpc\_Rqu-nlYZu\_-BFHNZec676\_BR/view?usp=sharing}{gifted.csv} da seguinte maneira.

\begin{minted}[frame=lines,fontsize=\scriptsize,linenos=false, bgcolor=LightGray]{r}
data <- read.csv('./data/csv/gifted.csv')
head(data,5)
\end{minted}

\begin{verbatim}
  score fatheriq motheriq speak count read edutv cartoons
1 159   115      117      18    26    1.9  3.00  2.00    
2 164   117      113      20    37    2.5  1.75  3.25    
3 154   115      118      20    32    2.2  2.75  2.50    
4 157   113      131      12    24    1.7  2.75  2.25    
5 156   110      109      17    34    2.2  2.25  2.50    
\end{verbatim}

\section{\href{https://r-data.pmagunia.com/dataset/r-dataset-package-mosaicdata-sat}{SAT - Professores}}
\label{sec:org7add5d8}
\begin{itemize}
\item \textbf{Modelar variável SAT (nota geral)}
\end{itemize}
\subsection{Sobre o \texttt{Conjunto}}
\label{sec:org35a3fdd}
\begin{itemize}
\item Dados:
\begin{itemize}
\item State: estado norte americano
\item Expend: termo médio de aula assistida diária.
\item Ratio: razão de alunos por professor.
\item Salary: salário do professor médio, anual.
\item Frac: percentagem de alunos elegíveis a participar do SAT.
\item Verbal: nota pra proficiência verbal no SAT.
\item Math: nota para proficiência em exatadas no SAT.
\item SAT: nota geral do SAT final.
\end{itemize}
\end{itemize}

\subsection{Como utilizar pelo R}
\label{sec:orge4ae9b7}

O \texttt{conjunto de dados} se encontra sob o pacote \texttt{mosaicData}. Desta forma, precisamos
instalá-lo.

\begin{minted}[frame=lines,fontsize=\scriptsize,linenos=false, bgcolor=LightGray]{r}
install.packages("mosaicData",mirror="https://vps.fmvz.usp.br/CRAN/")
\end{minted}

Após instalação, precisamos invocar o pacote,
\begin{minted}[frame=lines,fontsize=\scriptsize,linenos=false, bgcolor=LightGray]{r}
library(mosaicData)
\end{minted}

Finalmente, podemos acessar o \texttt{conjunto},
\begin{minted}[frame=lines,fontsize=\scriptsize,linenos=false, bgcolor=LightGray]{r}
data <- SAT
head(data)
\end{minted}

\subsection{Invocação de conjuto, por arquivo local}
\label{sec:org915e2ad}

Dado que você copiou os conjunto(s) de dado(s) para um diretório
local, pode-se chamar o conjunto \href{https://drive.google.com/file/d/1E6aOdH1nf0qF4Lky131iX1LNBjaIYAev/view?usp=sharing}{professores-SAT} da seguinte maneira.

\begin{minted}[frame=lines,fontsize=\scriptsize,linenos=false, bgcolor=LightGray]{r}
data <- read.csv('./data/csv/professores-SAT.csv')
head(data,5)
\end{minted}

\begin{verbatim}
  state      expend ratio salary frac verbal math sat 
1 Alabama    4.405  17.2  31.144  8   491    538  1029
2 Alaska     8.963  17.6  47.951 47   445    489   934
3 Arizona    4.778  19.3  32.175 27   448    496   944
4 Arkansas   4.459  17.1  28.934  6   482    523  1005
5 California 4.992  24.0  41.078 45   417    485   902
\end{verbatim}

\section{\href{https://r-data.pmagunia.com/dataset/r-dataset-package-robustbase-toxicity}{Toxicidade}}
\label{sec:org40e4d0c}
\begin{itemize}
\item \textbf{Modelar nível de toxicidade aquática de ácidos caboxílicos, baseado nos outros descritores moleculares.}
\end{itemize}
\subsection{Sobre o \texttt{Conjunto}}
\label{sec:org48b5e0e}
\begin{itemize}
\item Dados:
\begin{itemize}
\item toxicity
\end{itemize}
toxicidade  aquática, definida como o \(log(\textrm{IGC50}^{-1})\); tipicamente, a "reposta".
\begin{itemize}
\item logKow
log Kow, coeficiente de parcionamento
\item pKa
pKa: constante dissociativa
\item ELUMO
Energia do menor orbital molecular ocupado
\item Ecarb
Estado eletrotopolófico do grupo carboxílico
\item Emet
Estado eletrotopolófico do grupo metil
\item RM
Reflectibilidade Molar
\item IR
Index de Refração
\item Ts
Tensão superficial
\item P
Polaridade
\end{itemize}
\end{itemize}
\subsection{Como utilizar pelo R}
\label{sec:orgdd2a536}
O \texttt{conjunto de dados} se encontra sob o pacote \texttt{robustbase}. Desta forma, precisamos
instalá-lo.

\begin{minted}[frame=lines,fontsize=\scriptsize,linenos=false, bgcolor=LightGray]{r}
install.packages("robustbase",mirror="https://vps.fmvz.usp.br/CRAN/")
\end{minted}

Após instalação, precisamos invocar o pacote,
\begin{minted}[frame=lines,fontsize=\scriptsize,linenos=false, bgcolor=LightGray]{r}
library(robustbase)
\end{minted}

Finalmente, podemos acessar o \texttt{conjunto},
\begin{minted}[frame=lines,fontsize=\scriptsize,linenos=false, bgcolor=LightGray]{r}
data <- toxicity
head(data)
\end{minted}
\subsection{Invocação de conjuto, por arquivo local}
\label{sec:org8df62c1}

Dado que você copiou os conjunto(s) de dado(s) para um diretório
local, pode-se chamar o conjunto \href{https://drive.google.com/file/d/1dZE0wj0Z-FeXCvz-4ft0CPJ6Enn1fnew/view?usp=sharing}{toxicidade} da seguinte maneira.

\begin{minted}[frame=lines,fontsize=\scriptsize,linenos=false, bgcolor=LightGray]{r}
data <- read.csv('./data/csv/toxicity.csv')
head(data,5)
\end{minted}

\begin{verbatim}
  toxicity logKow pKa  ELUMO Ecarb   Emet   RM    IR    Ts   P    
1 -0.15    1.68   1.00 4.81  17.8635 1.4838 31.36 1.425 31.3 12.43
2 -0.33    0.94   0.98 4.68  16.9491 0.0000 22.10 1.408 30.4  8.76
3 -0.34    1.16   0.96 4.86  17.1806 0.2778 26.73 1.418 30.9 10.59
4  0.03    2.75   1.00 4.83  18.4794 3.5836 40.63 1.435 31.8 16.10
5 -0.57    0.79   0.97 4.80  16.8022 1.0232 22.14 1.411 32.5  8.77
\end{verbatim}

\section{\href{https://r-data.pmagunia.com/dataset/r-dataset-package-stat2data-fertility}{Fertilidade Feminina}}
\label{sec:org24d7a94}
\begin{itemize}
\item \textbf{Modelar MeanAFC} o qual diretamente empacta a habilidade de ter filhos.
\end{itemize}
\subsection{Sobre o \texttt{Conjunto}}
\label{sec:orgfaec0e1}
\begin{itemize}
\item Dados:
\begin{itemize}
\item Idade
\item LowAFC: Conta da menor quantidade de folículos antrais.
\item MeanAFC: Média de folículos antrais.
\item FSH: Máxima quantidade de níveis de hormônios estimulantes aos folículos.
\item E2: Nível de fertilidade.
\item MaxE2: Máximo nível de fertilidade.
\item MaxDailyGn: Nível máximo de gonadotrofina.
\item TotalGn: Nível total de gonadotrofina.
\item Oocytes: Quantidade de ovócitos/oócitos.
\item Embryos: Quantidade de embriões.
\end{itemize}
\end{itemize}
\subsection{Como utilizar pelo R}
\label{sec:org4ba49af}
O \texttt{conjunto de dados} se encontra sob o pacote \texttt{Stat2Data}. Desta forma, precisamos
instalá-lo.

\begin{minted}[frame=lines,fontsize=\scriptsize,linenos=false, bgcolor=LightGray]{r}
install.packages("Stat2Data",mirror="https://vps.fmvz.usp.br/CRAN/")
\end{minted}

Após instalação, precisamos invocar o pacote,
\begin{minted}[frame=lines,fontsize=\scriptsize,linenos=false, bgcolor=LightGray]{r}
library(Stat2Data)
\end{minted}

Finalmente, podemos acessar o \texttt{conjunto},
\begin{minted}[frame=lines,fontsize=\scriptsize,linenos=false, bgcolor=LightGray]{r}
data <- data("Fertility")
head(data)
\end{minted}
\subsection{Invocação de conjuto, por arquivo local}
\label{sec:org44743c6}

Dado que você copiou os conjunto(s) de dado(s) para um diretório
local, pode-se chamar o conjunto \href{https://drive.google.com/file/d/1iG15B5sUrntNJCjpIXT7Li1IHOlpvSKc/view?usp=sharing}{Fertilidade} da seguinte maneira.

\begin{minted}[frame=lines,fontsize=\scriptsize,linenos=false, bgcolor=LightGray]{r}
data <- read.csv('./data/csv/Fertility.csv')
head(data,5)
\end{minted}

\begin{verbatim}
  Age LowAFC MeanAFC FSH E2 MaxE2 MaxDailyGn TotalGn Oocytes Embryos
1 40  40     51.5    5.3 45 1427  300        2700    25      13     
2 37  41     41.0    7.1 53  802  225        1800     7       6     
3 40  38     41.0    4.9 40 4533  450        4850    27      15     
4 40  36     37.5    3.9 26 1804  300        2700     9       4     
5 30  36     36.0    4.0 49 2526  150        1500    19      12     
\end{verbatim}

\section{\href{https://r-data.pmagunia.com/dataset/r-dataset-package-datasets-attitude}{Atitudes em relação à empresa}}
\label{sec:orgfea1fe2}
\begin{itemize}
\item \textbf{Modelar e modelar o rating}
\end{itemize}
\subsection{Sobre o \texttt{Conjunto}}
\label{sec:org29595b6}
\begin{itemize}
\item Dados: (em percentagem)
\begin{itemize}
\item Complaints: percentagem de resoluções de reclamações.
\item Privileges: percentagem de intolerância de privilégios.
\item Learning: percentagem de oportunidade de aprendizado.
\item Raises: percetagem de aumento correspondente à perfôrmance.
\item Critical: percentagem de atitudes e críticas exacerbadas.
\item Advancel: percentadem de evolução percebida da empresa e posto.
\item Rating: avaliação geral positiva da empresa
\end{itemize}
\end{itemize}

\subsection{Como utilizar pelo R}
\label{sec:orgcfe5e00}
O \texttt{conjunto de dados} se encontra sob o pacote \texttt{datasets}. Desta forma, precisamos
instalá-lo.

\begin{minted}[frame=lines,fontsize=\scriptsize,linenos=false, bgcolor=LightGray]{r}
install.packages("datasets",mirror="https://vps.fmvz.usp.br/CRAN/")
\end{minted}

Após instalação, precisamos invocar o pacote,
\begin{minted}[frame=lines,fontsize=\scriptsize,linenos=false, bgcolor=LightGray]{r}
library(datasets)
\end{minted}

Finalmente, podemos acessar o \texttt{conjunto},
\begin{minted}[frame=lines,fontsize=\scriptsize,linenos=false, bgcolor=LightGray]{r}
data <- data("attitude")
head(data)
\end{minted}
\subsection{Invocação de conjuto, por arquivo local}
\label{sec:org2caac3c}

Dado que você copiou os conjunto(s) de dado(s) para um diretório
local, pode-se chamar o conjunto \href{https://drive.google.com/file/d/1rKj4NPD61bWKD6HBC4fux2Eit6CNEKwr/view?usp=sharing}{Atitude} da seguinte maneira.

\begin{minted}[frame=lines,fontsize=\scriptsize,linenos=false, bgcolor=LightGray]{r}
data <- read.csv('./data/csv/attitude.csv')
head(data,5)
\end{minted}

\begin{verbatim}
  rating complaints privileges learning raises critical advance
1 43     51         30         39       61     92       45     
2 63     64         51         54       63     73       47     
3 71     70         68         69       76     86       48     
4 61     63         45         47       54     84       35     
5 81     78         56         66       71     83       47     
\end{verbatim}

\section{\href{https://r-data.pmagunia.com/dataset/r-dataset-package-datasets-lifecyclesavings}{Hipótese de ciclos-de-economia salarial}}
\label{sec:orgf57994f}
\begin{itemize}
\item \textbf{Modelar \texttt{sr}, a partir das outra variáveis}
\item Hipótese formulada por Franco Modigliani 1960-1970, de que essas (outras)
variáveis eram explicativas do fenômeno 'sr'.
\end{itemize}
\subsection{Sobre o \texttt{Conjunto}}
\label{sec:orgd4e28b7}
\begin{itemize}
\item Dados:
\begin{itemize}
\item Sr: valor agregado à economia particular (razão entre valor total de economias pessoais e salário líquido)
\item Pop15: população sob quinze anos de idade.
\item Pop75: população acima de setenta e cinco anos de idade.
\item dpi: valor de salário líquido per-capita médio.
\item ddpi: taxa de crescimento de dpi.
\end{itemize}
\end{itemize}

\subsection{Como utilizar pelo R}
\label{sec:org0dc5e41}
O \texttt{conjunto de dados} se encontra sob o pacote \texttt{datasets}. Desta forma, precisamos
instalá-lo.

\begin{minted}[frame=lines,fontsize=\scriptsize,linenos=false, bgcolor=LightGray]{r}
install.packages("datasets",mirror="https://vps.fmvz.usp.br/CRAN/")
\end{minted}

Após instalação, precisamos invocar o pacote,
\begin{minted}[frame=lines,fontsize=\scriptsize,linenos=false, bgcolor=LightGray]{r}
library(datasets)
\end{minted}

Finalmente, podemos acessar o \texttt{conjunto},
\begin{minted}[frame=lines,fontsize=\scriptsize,linenos=false, bgcolor=LightGray]{r}
data <- data("LifeCycleSavings")
head(data)
\end{minted}
\subsection{Invocação de conjuto, por arquivo local}
\label{sec:org2896cb9}

Dado que você copiou os conjunto(s) de dado(s) para um diretório
local, pode-se chamar o conjunto \href{https://drive.google.com/file/d/1j2K7J1rb3V2Qr\_t0rcBhA6tyuqh88AjY/view?usp=sharing}{savings.csv} da seguinte maneira.

\begin{minted}[frame=lines,fontsize=\scriptsize,linenos=false, bgcolor=LightGray]{r}
data <- read.csv('./data/csv/savings.csv')
head(data,5)
\end{minted}

\begin{verbatim}
          sr    pop15 pop75 dpi     ddpi
Australia 11.43 29.35 2.87  2329.68 2.87
Austria   12.07 23.32 4.41  1507.99 3.93
Belgium   13.17 23.80 4.43  2108.47 3.82
Bolivia    5.75 41.89 1.67   189.13 0.22
Brazil    12.88 42.19 0.83   728.47 4.56
\end{verbatim}

\section{\href{https://r-data.pmagunia.com/dataset/r-dataset-package-ecdat-computers}{Preço de Computadores 1993-1995}}
\label{sec:orgde96e65}
\begin{itemize}
\item \textbf{Modelar preço (\texttt{price}) a partir das variáveis}.
\end{itemize}
\subsection{Dados sobre o \texttt{Conjunto}}
\label{sec:org5b0fc5a}
\begin{itemize}
\item Dados:
\begin{itemize}
\item Speed: velocidade de rotação em MHz do processador (clock speed)
\item HD: tamanho da memória do disco rígido em MB
\item RAM: tamanho da RAM em MB
\item Screen: tamanho da tela em polegadas
\item CD: tem ou não entrada para CDs.
\item Multi: kit multimídia incluso ou não (caixa de som etc).
\item Premium: manufatura feita por compania conhecida (IBM etc).
\item Ads: quantidade de anúncio do computador por mês.
\item Trend: quanto tempo está no mercado.
\end{itemize}
\end{itemize}
\subsection{Como utilizar pelo R}
\label{sec:orgbd1fec3}
O \texttt{conjunto de dados} se encontra sob o pacote \texttt{Ecdat}. Desta forma, precisamos
instalá-lo.

\begin{minted}[frame=lines,fontsize=\scriptsize,linenos=false, bgcolor=LightGray]{r}
install.packages("Ecdat",mirror="https://vps.fmvz.usp.br/CRAN/")
\end{minted}

Após instalação, precisamos invocar o pacote,
\begin{minted}[frame=lines,fontsize=\scriptsize,linenos=false, bgcolor=LightGray]{r}
library(Ecdat)
\end{minted}

Finalmente, podemos acessar o \texttt{conjunto},
\begin{minted}[frame=lines,fontsize=\scriptsize,linenos=false, bgcolor=LightGray]{r}
data <- data("Computers")
head(data)
\end{minted}

\subsection{Invocação de conjuto, por arquivo local}
\label{sec:orgffce1ec}

Dado que você copiou os conjunto(s) de dado(s) para um diretório
local, pode-se chamar o conjunto \href{https://drive.google.com/file/d/1C1-9aM-dYzx7UQrWq\_gg4DRw6wtj-l7Z/view?usp=sharing}{computers.csv} da seguinte maneira.

\begin{minted}[frame=lines,fontsize=\scriptsize,linenos=false, bgcolor=LightGray]{r}
data <- read.csv('./data/csv/computers.csv')
head(data,5)
\end{minted}

\begin{verbatim}
  price speed hd  ram screen cd multi premium ads trend
1 1499  25     80  4  14     no no    yes     94  1    
2 1795  33     85  2  14     no no    yes     94  1    
3 1595  25    170  4  15     no no    yes     94  1    
4 1849  25    170  8  14     no no    no      94  1    
5 3295  33    340 16  14     no no    yes     94  1    
\end{verbatim}


\section{\href{https://www.sheffield.ac.uk/mash/statistics/datasets}{Peso ao Nascimento}}
\label{sec:org2845403}
\begin{itemize}
\item \textbf{Modelar peso ao nascimento (libras) (\texttt{Birthweight}) a partir das variáveis}.
\end{itemize}
\subsection{Dados sobre o \texttt{Conjunto}}
\label{sec:orga4d228d}
\begin{itemize}
\item Dados:
\begin{itemize}
\item lowbwt: baixo peso ao nascer.
\item ID: Identificador
\item mage35: mãe mais velha do que 35 anos.
\item fnocig: número de cigarros fumado pelo pai diariamente.
\item fheight: peso do pai (kg)
\item fedyrs: quantidade de anos de educação do pai
\item fage: idade do pai
\item mppwt: peso da mãe pré gravidez.
\item mheight: tamanho da mãe (cm).
\item mnocig: número de cigarros fumados diariamente pela mãe.
\item Motherage: idade da mãe.
\item Gestation Smoker: fumante durante gestação
\item headcirumfer: circunferência da cabeça
\item length: tamanho (cm)
\end{itemize}
\end{itemize}
\subsection{Invocação de conjuto, por arquivo local}
\label{sec:org75e9aa6}

Dado que você copiou os conjunto(s) de dado(s) para um diretório
local, pode-se chamar o conjunto \href{https://drive.google.com/file/d/1xv2lCPsj04FjGPQ\_BgPS9mrTIjBcyHQk/view?usp=sharing}{Birth Weight} da seguinte maneira.

\begin{minted}[frame=lines,fontsize=\scriptsize,linenos=false, bgcolor=LightGray]{r}
data <- read.csv('./data/csv/Birthweight_reduced_kg_R.csv')
head(data,5)
\end{minted}

\begin{verbatim}
  X...ID Length Birthweight Headcirc Gestation smoker mage mnocig mheight mppwt
1 1360   56     4.55        34       44        0      20   0      162     57   
2 1016   53     4.32        36       40        0      19   0      171     62   
3  462   58     4.10        39       41        0      35   0      172     58   
4 1187   53     4.07        38       44        0      20   0      174     68   
5  553   54     3.94        37       42        0      24   0      175     66   
  fage fedyrs fnocig fheight lowbwt mage35
1 23   10     35     179     0      0     
2 19   12      0     183     0      0     
3 31   16     25     185     0      1     
4 26   14     25     189     0      0     
5 30   12      0     184     0      0     
\end{verbatim}

\section{\href{https://www.sheffield.ac.uk/mash/statistics/datasets}{Crime}}
\label{sec:orgb32a1b8}
\begin{itemize}
\item \textbf{Modelar taxa de crime (\texttt{CrimeRate}) a partir das variáveis}.
\end{itemize}
\subsection{Dados sobre o \texttt{Conjunto}}
\label{sec:orga7c524d}
\begin{itemize}
\item Dados:
\begin{itemize}
\item CrimeRate: Taxa de criminalidade (ofensas por milhão de habitantes).
\item Youth: Jovens entre 18-24 anos a cada mil habitantes.
\item Southern: Estado do sul?
\item Education: Tempo de educação (anos de estudo até 25)
\item ExpenditureYear0: Dinheiro para segurança pública per capta.
\item LabourForce: Quantidade de jovens trabalhadores por 1000 habitantes.
\item Males: Homens (por cada 1000 mulheres).
\item MoreMales: Mais homens que mulheres?.
\item StateSize: Tamanho do estado em milhares.
\item YouthUnemployment: Desemprego de jovens por mil.
\item BelowWage: Número de famílias abaixo de meio salário por mil.
\item Wage: Salário médio semanal.
\item MatureUnemployment: Desemprego de sêniors (35-39) por mil.
\end{itemize}
\end{itemize}

\subsection{Invocação de conjuto, por arquivo local}
\label{sec:org442b85f}

Dado que você copiou os conjunto(s) de dado(s) para um diretório
local, pode-se chamar o conjunto \href{https://drive.google.com/file/d/1hZpHoEXbhZGvmtmrbcWpXYfBV-2ZD7uF/view?usp=sharing}{Crimes} da seguinte maneira.

\begin{minted}[frame=lines,fontsize=\scriptsize,linenos=false, bgcolor=LightGray]{r}
data <- read.csv('./data/csv/Crime_R.csv')
head(data,5)
\end{minted}

\begin{verbatim}
  X...CrimeRate Youth Southern Education ExpenditureYear0 LabourForce Males
1 45.5          135   0        12.4       69              540          965 
2 52.3          140   0        10.9       55              535         1045 
3 56.6          157   1        11.2       47              512          962 
4 60.3          139   1        11.9       46              480          968 
5 64.2          126   0        12.2      106              599          989 
  MoreMales StateSize YouthUnemployment ... ExpenditureYear10 LabourForce10
1 0          6         80               ... 71                564          
2 1          6        135               ... 54                540          
3 0         22         97               ... 44                529          
4 0         19        135               ... 41                497          
5 0         40         78               ... 97                602          
  Males10 MoreMales10 StateSize10 YouthUnemploy10 MatureUnemploy10
1  974    0            6           82             20              
2 1039    1            7          138             39              
3  959    0           24           98             33              
4  983    0           20          131             50              
5  989    0           42           79             24              
  HighYouthUnemploy10 Wage10 BelowWage10
1 1                   632    142        
2 1                   521    210        
3 0                   359    256        
4 0                   510    235        
5 1                   660    162        
\end{verbatim}

\section{\href{https://archive.ics.uci.edu/ml/datasets/Student+Performance\#}{Performance de Estudantes}}
\label{sec:org75078c3}
\begin{itemize}
\item \textbf{Modelar nota final (\texttt{G3}) a partir das variáveis}.
\end{itemize}
\subsection{Dados sobre o \texttt{Conjunto}}
\label{sec:orgf4e6b51}
\begin{itemize}
\item Dados:
\begin{itemize}
\item school - Escola em que estuda (binary: 'GP' - Gabriel Pereira ou 'MS' - Mousinho da Silveira)
\item sex - Sexo (binary: 'F' - mulher ou 'M' - homem)
\item age - Idade (numeric: from 15 to 22)
\item address - Estadia/tipo da região (binary: 'U' - urbana or 'R' - rural)
\item famsize - Tamanho familhar (binary: 'LE3' - menos que  3 or 'GT3' - mais que 3)
\item Pstatus - Estado de co-habitação dos pais (binary: 'T' - junto or 'A' - aparte)
\item Medu - Educação da mãe (numeric: 0 - nenhuma, 1 - primária (4th ano), 2 - 5th ao 9th ano, 3 - secundária, ou 4 - superior)
\item Fedu - Educação do pai (numeric: 0 - nenhuma, 1 - primária (4th ano), 2 - 5th ao 9th ano, 3 - secundária, ou 4 - superior)
\item Mjob - Tabalho da mãe (nominal: 'teacher', 'health' care related, civil 'services' (e.g. administrative or police), 'at\textsubscript{home}' or 'other')
\item Fjob - Trabalho do pai (nominal: 'teacher', 'health' care related, civil 'services' (e.g. administrative or police), 'at\textsubscript{home}' or 'other')
\item reason - Razão de escolha da escola (nominal: close to 'home', school 'reputation', 'course' preference or 'other')
\item guardian - Guardião legal (nominal: 'mother', 'father' or 'other')
\item traveltime - Tempo de viagem, escola à casa (numeric: 1 - <15 min., 2 - 15 to 30 min., 3 - 30 min. to 1 hour, or 4 - >1 hour)
\item studytime - Tempo de estudo semanal (numeric: 1 - <2 hours, 2 - 2 to 5 hours, 3 - 5 to 10 hours, or 4 - >10 hours)
\item failures - Número de reprovações (numeric: n if 1<=n<3, else 4)
\item schoolsup - Suporte educacional extra (binary: yes or no)
\item famsup - Suporte educacional familiar (binary: yes or no)
\item paid - Classes extras pagas (Math or Portuguese) (binary: yes or no)
\item activities - Atividades extra-curriculares (binary: yes or no)
\item nursery - Atendeu a clases de enfermaria (binary: yes or no)
\item higher - Quer seguir educação superior (binary: yes or no)
\item internet - Acesso a internet (binary: yes or no)
\item romantic - Em relação romántica (binary: yes or no)
\item famrel - Qualidade de relação familiar (numeric: from 1 - very bad to 5 - excellent)
\item freetime - Tempo livre fora da escola (numeric: from 1 - very low to 5 - very high)
\item goout - Tempo gasto saindo com amigos (numeric: from 1 - very low to 5 - very high)
\item Dalc - Consumo diário de álcool (numeric: from 1 - very low to 5 - very high)
\item Walc - Consumo em finais de semana de alcool (numeric: from 1 - very low to 5 - very high)
\item health - Estado de saúde atual (numeric: from 1 - very bad to 5 - very good)
\item absences - Número de faltas (numeric: from 0 to 93)
\item G1 - Notas primeiro período (numeric: from 0 to 20) [relacionadas com português ou matemática]
\item G2 - Nota segundo período (numeric: from 0 to 20) [relacionadas com português ou matemática]
\end{itemize}
\end{itemize}

\begin{itemize}
\item G3 - Nota final (numeric: from 0 to 20, output target) [relacionadas com português ou matemática]
\end{itemize}
\subsection{Invocação de conjuto, por arquivo local}
\label{sec:orge4d3441}

Dado que você copiou os conjunto(s) de dado(s) para um diretório
local, pode-se chamar o conjunto de métricas e notas de \href{https://drive.google.com/file/d/1vrIpfxXLxqo4ngPDOBjyrJFhB-Ei5N4C/view?usp=sharing}{Português} e \href{https://drive.google.com/file/d/1FO2LEQhKbCX8cq6yBxqKzjrbnR18OxSD/view?usp=sharing}{Matemática} da seguinte maneira,

\begin{minted}[frame=lines,fontsize=\scriptsize,linenos=false, bgcolor=LightGray]{r}
data_mat=read.table("./data/csv/student-mat.csv",sep=";",header=TRUE)
data_por=read.table("./data/csv/student-por.csv",sep=";",header=TRUE)

data=merge(data_mat,data_por,by=c("school","sex","age","address","famsize","Pstatus","Medu",
"Fedu","Mjob","Fjob","reason","nursery","internet"))
print(nrow(data)) # 382 dados

head(data,5)
\end{minted}

\begin{verbatim}
[1] 382
  school sex age address famsize Pstatus Medu Fedu Mjob     Fjob     ...
1 GP     F   15  R       GT3     T       1    1    at_home  other    ...
2 GP     F   15  R       GT3     T       1    1    other    other    ...
3 GP     F   15  R       GT3     T       2    2    at_home  other    ...
4 GP     F   15  R       GT3     T       2    4    services health   ...
5 GP     F   15  R       GT3     T       3    3    services services ...
  famrel.y freetime.y goout.y Dalc.y Walc.y health.y absences.y G1.y G2.y G3.y
1 3        1          2       1      1      1        4          13   13   13  
2 3        3          4       2      4      5        2          13   11   11  
3 4        3          1       1      1      2        8          14   13   12  
4 4        3          2       1      1      5        2          10   11   10  
5 4        2          1       2      3      3        2          13   13   13  
\end{verbatim}

\begin{minted}[frame=lines,fontsize=\scriptsize,linenos=false, bgcolor=LightGray]{r}
summary(data)
\end{minted}

\begin{verbatim}
   school              sex                 age          address         
Length:382         Length:382         Min.   :15.00   Length:382        
Class :character   Class :character   1st Qu.:16.00   Class :character  
Mode  :character   Mode  :character   Median :17.00   Mode  :character  
                                      Mean   :16.59                     
                                      3rd Qu.:17.00                     
                                      Max.   :22.00                     
  famsize            Pstatus               Medu            Fedu      
Length:382         Length:382         Min.   :0.000   Min.   :0.000  
Class :character   Class :character   1st Qu.:2.000   1st Qu.:2.000  
Mode  :character   Mode  :character   Median :3.000   Median :3.000  
                                      Mean   :2.806   Mean   :2.565  
                                      3rd Qu.:4.000   3rd Qu.:4.000  
                                      Max.   :4.000   Max.   :4.000  
    Mjob               Fjob              reason            nursery         
Length:382         Length:382         Length:382         Length:382        
Class :character   Class :character   Class :character   Class :character  
Mode  :character   Mode  :character   Mode  :character   Mode  :character  



  internet          guardian.x         traveltime.x    studytime.x   
Length:382         Length:382         Min.   :1.000   Min.   :1.000  
Class :character   Class :character   1st Qu.:1.000   1st Qu.:1.000  
Mode  :character   Mode  :character   Median :1.000   Median :2.000  
                                      Mean   :1.442   Mean   :2.034  
                                      3rd Qu.:2.000   3rd Qu.:2.000  
                                      Max.   :4.000   Max.   :4.000  
  failures.x     schoolsup.x          famsup.x            paid.x         
Min.   :0.0000   Length:382         Length:382         Length:382        
1st Qu.:0.0000   Class :character   Class :character   Class :character  
Median :0.0000   Mode  :character   Mode  :character   Mode  :character  
Mean   :0.2906                                                           
3rd Qu.:0.0000                                                           
Max.   :3.0000                                                           
activities.x         higher.x          romantic.x           famrel.x   
Length:382         Length:382         Length:382         Min.   :1.00  
Class :character   Class :character   Class :character   1st Qu.:4.00  
Mode  :character   Mode  :character   Mode  :character   Median :4.00  
                                                         Mean   :3.94  
                                                         3rd Qu.:5.00  
                                                         Max.   :5.00  
  freetime.x       goout.x          Dalc.x          Walc.x        health.x    
Min.   :1.000   Min.   :1.000   Min.   :1.000   Min.   :1.00   Min.   :1.000  
1st Qu.:3.000   1st Qu.:2.000   1st Qu.:1.000   1st Qu.:1.00   1st Qu.:3.000  
Median :3.000   Median :3.000   Median :1.000   Median :2.00   Median :4.000  
Mean   :3.223   Mean   :3.113   Mean   :1.474   Mean   :2.28   Mean   :3.579  
3rd Qu.:4.000   3rd Qu.:4.000   3rd Qu.:2.000   3rd Qu.:3.00   3rd Qu.:5.000  
Max.   :5.000   Max.   :5.000   Max.   :5.000   Max.   :5.00   Max.   :5.000  
  absences.x          G1.x            G2.x            G3.x      
Min.   : 0.000   Min.   : 3.00   Min.   : 0.00   Min.   : 0.00  
1st Qu.: 0.000   1st Qu.: 8.00   1st Qu.: 8.25   1st Qu.: 8.00  
Median : 3.000   Median :10.50   Median :11.00   Median :11.00  
Mean   : 5.319   Mean   :10.86   Mean   :10.71   Mean   :10.39  
3rd Qu.: 8.000   3rd Qu.:13.00   3rd Qu.:13.00   3rd Qu.:14.00  
Max.   :75.000   Max.   :19.00   Max.   :19.00   Max.   :20.00  
 guardian.y         traveltime.y    studytime.y      failures.y    
Length:382         Min.   :1.000   Min.   :1.000   Min.   :0.0000  
Class :character   1st Qu.:1.000   1st Qu.:1.000   1st Qu.:0.0000  
Mode  :character   Median :1.000   Median :2.000   Median :0.0000  
                   Mean   :1.445   Mean   :2.039   Mean   :0.1414  
                   3rd Qu.:2.000   3rd Qu.:2.000   3rd Qu.:0.0000  
                   Max.   :4.000   Max.   :4.000   Max.   :3.0000  
schoolsup.y          famsup.y            paid.y          activities.y      
Length:382         Length:382         Length:382         Length:382        
Class :character   Class :character   Class :character   Class :character  
Mode  :character   Mode  :character   Mode  :character   Mode  :character  



  higher.y          romantic.y           famrel.y       freetime.y  
Length:382         Length:382         Min.   :1.000   Min.   :1.00  
Class :character   Class :character   1st Qu.:4.000   1st Qu.:3.00  
Mode  :character   Mode  :character   Median :4.000   Median :3.00  
                                      Mean   :3.942   Mean   :3.23  
                                      3rd Qu.:5.000   3rd Qu.:4.00  
                                      Max.   :5.000   Max.   :5.00  
   goout.y          Dalc.y          Walc.y         health.y    
Min.   :1.000   Min.   :1.000   Min.   :1.000   Min.   :1.000  
1st Qu.:2.000   1st Qu.:1.000   1st Qu.:1.000   1st Qu.:3.000  
Median :3.000   Median :1.000   Median :2.000   Median :4.000  
Mean   :3.118   Mean   :1.476   Mean   :2.291   Mean   :3.576  
3rd Qu.:4.000   3rd Qu.:2.000   3rd Qu.:3.000   3rd Qu.:5.000  
Max.   :5.000   Max.   :5.000   Max.   :5.000   Max.   :5.000  
  absences.y          G1.y            G2.y            G3.y      
Min.   : 0.000   Min.   : 0.00   Min.   : 5.00   Min.   : 0.00  
1st Qu.: 0.000   1st Qu.:10.00   1st Qu.:11.00   1st Qu.:11.00  
Median : 2.000   Median :12.00   Median :12.00   Median :13.00  
Mean   : 3.673   Mean   :12.11   Mean   :12.24   Mean   :12.52  
3rd Qu.: 6.000   3rd Qu.:14.00   3rd Qu.:14.00   3rd Qu.:14.00  
Max.   :32.000   Max.   :19.00   Max.   :19.00   Max.   :19.00  
\end{verbatim}
\end{document}